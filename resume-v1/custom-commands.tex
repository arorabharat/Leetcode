%-------------------------%
% Custom Commands
%-------------------------%

%-------------------------%
% List Item Commands
%-------------------------%

% Begins a bulleted list with circular bullets (•)
% Must be paired with \resumeItemMiniPageListEnd
% Usage: \resumeItemMiniPageListStart ... \resumeItemMiniPageListEnd
\newcommand{\resumeItemMiniPageListStart}{\begin{itemize}[label=\textbullet]}

% Ends a bulleted list and adds spacing
% Must be paired with \resumeItemMiniPageListStart
% Usage: \resumeItemMiniPageListStart ... \resumeItemMiniPageListEnd
\newcommand{\resumeItemMiniPageListEnd}{\end{itemize}\vspace{-5pt}}

% Ensures second-level nested bullets are also circular (tiny •)
% This maintains consistent bullet styling throughout the document
\renewcommand{\labelitemii}{$\vcenter{\hbox{\tiny$\bullet$}}$}

%-------------------------%
% Section List Commands
%-------------------------%

%-------------------------%
% Heading Commands
%-------------------------%

% Creates an experience entry heading with company, role, and date
% Usage: \resumeExperienceHeading{Company Name}{Job Title}{Date Range}
% Parameters: [1] - Company name (bold)
%            [2] - Job title/role (small, in parentheses)
%            [3] - Date range (small, italic, right-aligned)
% Example: \resumeExperienceHeading{Amazon}{Software Engineer}{Jan 2020 -- Present}
\newcommand{\resumeExperienceHeading}[3]{
    \vspace{4pt}\item
    \begin{tabular*}{0.96\textwidth}[t]{l@{\extracolsep{\fill}}r}
    \textbf{#1} (\small{#2})  & \textit{\small{#3}} \vspace{4pt}
    \end{tabular*}\vspace{-7pt}
}


%-------------------------%
% Icon Commands
%-------------------------%

% Renders a FontAwesome icon in black color
% Usage: \resumeIcon{faIconName}
% Parameters: [1] - FontAwesome icon name (without the 'fa' prefix in the command)
% Example: \resumeIcon{faPhone} renders the phone icon
% Note: Requires fontawesome5 package and the icon must be defined
\newcommand{\resumeIcon}[1]{\textcolor{black}{\csname #1\endcsname}}

%-------------------------%
% Metric Formatting Commands
%-------------------------%

% Formats a percentage metric (e.g., 3% becomes ~3%)
% Usage: \metricPercent{3} produces ~3%
% Parameters: [1] - The percentage value (number only, without % sign)
\newcommand{\metricPercent}[1]{$\sim#1\%$}

% Formats an approximate metric with unit (e.g., 620K, 6M)
% Usage: \metricApprox{620K} produces ~620K
% Parameters: [1] - The metric value with unit (e.g., 620K, 6M, 80K)
\newcommand{\metricApprox}[1]{$\sim#1$}

% Formats a range metric (e.g., 4-5M becomes ~4-5M)
% Usage: \metricRange{4-5M} produces ~4-5M
% Parameters: [1] - The range value with unit (e.g., 4-5M, 1.2-1.5K)
\newcommand{\metricRange}[1]{$\sim#1$}

% Formats a greater-than metric (e.g., >80%)
% Usage: \metricGreater{80} produces >80%
% Parameters: [1] - The value (number only, without % sign)
\newcommand{\metricGreater}[1]{$>#1\%$}

% Formats a greater-than metric with unit (e.g., >100K)
% Usage: \metricGreaterUnit{100K} produces >100K
% Parameters: [1] - The value with unit (e.g., 100K, 50M)
\newcommand{\metricGreaterUnit}[1]{$>#1$}

%-------------------------%
% Text Formatting Commands
%-------------------------%

% Makes text bold (for highlighting key achievements, technologies, etc.)
% Usage: \resumeBold{text to make bold}
% Parameters: [1] - Text to make bold
% Example: \resumeBold{key achievement} produces bold text
\newcommand{\resumeBold}[1]{\textbf{#1}}

% Creates a clickable link with optional bold formatting
% Usage: \resumeLink{URL}{Link Text}
% Parameters: [1] - URL for the link
%            [2] - Text to display (will be bold)
% Example: \resumeLink{https://example.com}{Example} creates a bold clickable link
\newcommand{\resumeLink}[2]{\href{#1}{\resumeBold{#2}}}

% Formats the resume name (large, small caps, bold)
% Usage: \resumeName{Name}
% Parameters: [1] - Name to display
% Example: \resumeName{Bharat Arora}
\newcommand{\resumeName}[1]{\textbf{\Huge \scshape #1}}

%-------------------------%
% Spacing Commands
%-------------------------%

% Horizontal spacing between contact items
% Usage: \resumeContactSpacing
% Note: Provides consistent spacing between contact information items
\newcommand{\resumeContactSpacing}{\qquad}

% Vertical spacing before first item in a section
% Usage: \resumeItemMiniPageSpacing
% Note: Provides consistent 4pt spacing before section items
\newcommand{\resumeItemMiniPageSpacing}{\vspace{4pt}}

% Vertical spacing after name in header
% Usage: \resumeNameSpacing
% Note: Provides consistent 10pt spacing after the name
\newcommand{\resumeNameSpacing}{\vspace{10pt}}

% Line break with spacing
% Usage: \resumeLineBreak
% Note: Creates a line break
\newcommand{\resumeLineBreak}{\\}

% Line break with extra spacing (for skills section)
% Usage: \resumeLineBreakSpaced
% Note: Creates a line break with 0.5ex spacing
\newcommand{\resumeLineBreakSpaced}{\\[0.5ex]}

%-------------------------%
% Contact Information Commands
%-------------------------%

% Creates a contact information item with icon and text (for non-linked items like phone)
% Usage: \resumeContactItem{faIconName}{Contact Text}
% Parameters: [1] - FontAwesome icon name
%            [2] - Contact information text
% Example: \resumeContactItem{faPhone}{+91-9673213072}
\newcommand{\resumeContactItem}[2]{%
    \resumeIcon{#1} \small{#2}%
}

% Creates a contact information link with icon and text (for linked items like email, LinkedIn)
% Usage: \resumeContactLink{URL}{faIconName}{Link Text}
% Parameters: [1] - URL for the link
%            [2] - FontAwesome icon name
%            [3] - Link text to display
% Example: \resumeContactLink{mailto:email@example.com}{faEnvelope}{email@example.com}
\newcommand{\resumeContactLink}[3]{%
    \href{#1}{\small{\resumeIcon{#2} #3}}%
}

% =================================================================

% Custom commands
\newcommand{\resumeItem}[1]{
    \item\small{#1}
}

\newcommand{\resumeItemMiniPage}[1]{%
    \item
    \begin{minipage}[t]{0.97\linewidth}
    #1
    \end{minipage}%
}


\newcommand{\resumeSubheading}[4]{
    \vspace{-2pt}\item
    \begin{tabular*}{0.97\textwidth}[t]{l@{\extracolsep{\fill}}r}
    \textbf{#1} & #2 \\
    \textit{\small#3} & \textit{\small #4} \\
    \end{tabular*}\vspace{-7pt}
}

\newcommand{\resumeSubSubheading}[2]{
    \item
    \begin{tabular*}{0.97\textwidth}{l@{\extracolsep{\fill}}r}
    \textit{\small#1} & \textit{\small #2} \\
    \end{tabular*}\vspace{-7pt}
}

\newcommand{\resumeProjectHeading}[2]{
    \item
    \begin{tabular*}{0.97\textwidth}{l@{\extracolsep{\fill}}r}
    \small#1 & #2 \\
    \end{tabular*}\vspace{-7pt}
}

\newcommand{\resumeSubItem}[1]{\resumeItem{#1}\vspace{-4pt}}

\renewcommand\labelitemii{$\vcenter{\hbox{\tiny$\bullet$}}$}



\newcommand{\resumeItemListStart}{\begin{itemize}}
\newcommand{\resumeItemListEnd}{\end{itemize}\vspace{-5pt}}

\definecolor{Black}{RGB}{0, 0, 0}
\newcommand{\seticon}[1]{\textcolor{Black}{\csname #1\endcsname}}

% reviewed

\newcommand{\arrangeInTwoByTwoTable}[4]{
    \vspace{4pt}\item
    \begin{tabular*}{0.96\textwidth}[t]{l@{\extracolsep{\fill}}r}
    \textbf{#1} & \textit{\small{#2}} \\[0.5ex]
    \textit{#3} & \textit{\small{#4}}
    \end{tabular*}
}

\newcommand{\resumeSubheadingExperience}[3]{
    \vspace{4pt}\item
    \begin{tabular*}{0.96\textwidth}[t]{l@{\extracolsep{\fill}}r}
    \textbf{#1} (\small{#2})  & \textit{\small{#3}} \vspace{4pt}
    \end{tabular*}\vspace{-7pt}
}

% heading under all sections will be 0.1 inch away from section vertical line
\newcommand{\startListPointOneInchAwayFromLeft}{\begin{itemize}[leftmargin=0.1in, label={}]}
\newcommand{\makeLinePointOneInchAwayEnd}{\end{itemize}}
% ===============